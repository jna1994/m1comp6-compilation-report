\documentclass[a4paper,12pt]{article}
\usepackage[utf8x]{inputenc}
\usepackage[french]{babel}
\usepackage{graphicx}
\usepackage{float}

\title{Compilation de MiniJaja vers le JajaCode}

\begin{document}
\maketitle

Dans cette partie du rapport la discussion porte sur les modules  « JajaCode » et « Compiler ». Premier contient des classes nécessaires pour la construction d’un objet JajaCode et deuxième c’est un visiteur Minjaja, qui prend à l’entrée l’arbre MiniJaja et sort le Jajacode correspondant. 

\section{Conception et traitement des erreurs}
Pour la construction de JajaCode le pattern « builder » a été mis en place. JajaCodeBuilder est constructeur de JajaCode qui contient une liste des instructions de JajaCode. Le compiler les remplie au fur à mesure. Une fois les instructions de JajaCode sont construits, le methode build() généré l’objet JajaCode qui contient la séquence des instructions JajaCodede.
  
De plus, pour rendre les instructions de JajaCode compréhensible aux utilisateurs (au format chaîne de caractère) la classe JajaCodeRenderer mis en place . JajaCodeRenderer utilise le pattern visiteur et  hérite de l’interface JajaCodeInstructions. 

MiniJajaCompiler consiste de deux visiteurs. La classe MiniJajaCompiler et sa classe privée  RevokeVisitor. Le visiteur RevokeVisitor est dédié au  retrait de déclaration. 

MiniJajaCompiler supposé de recevoir à l’entré le code source correct de MiniJaja, cela veut dire que avant de passé par compiler l’analyseur effectue l’analyse lexicale et syntaxique et contrôleur de type vérifier le compatibilité des typages, l’adéquation des valeurs. Alors le MiniJajaCompiler ne gère pas des erreurs. 

\section{Tests}
La classe test MiniJajaCompilerTest teste la compilation du code source MiniJaja vers JajaCode. Pour cela les exemples de support de cours et des travaux dirigé sont utilisés.  On donne à l’entrée MiniJajaCompiler un code source MiniJaja et reçoit en sortie l’objet JajaCode.  Ensuite on convertit chaque instruction de JajaCode obtenu aux chaînes de caractères à l’aide JajaCodeRenderer. Alors on vérifier bien que compilation MiniJaja correspond bien à JajaCode que nous attend. 
\end{document}
