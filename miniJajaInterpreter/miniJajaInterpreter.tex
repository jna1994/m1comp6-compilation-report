\documentclass[a4paper,12pt]{article}
\usepackage[utf8x]{inputenc}
\usepackage[french]{babel}
\usepackage{graphicx}
\usepackage{listings}
\usepackage[T1]{fontenc}
\usepackage{color}

\definecolor{keyword_color}{rgb}{0.13,0.13,1}
\definecolor{string_color}{rgb}{0,0.5,0}
\definecolor{comment_color}{rgb}{0.46,0.45,0.48}

\usepackage{listings}
\lstset{language=Java,
  basicstyle=\footnotesize\ttfamily,
  showspaces=false,
  showtabs=false,
  breaklines=true,
  showstringspaces=false,
  breakatwhitespace=true,
  commentstyle=\color{comment_color}\emph,
  keywordstyle=\color{keyword_color},
  stringstyle=\color{string_color},
  frame=single,
  literate=%
         {à}{{\`a}}1
         {é}{{\'e}}1
         {è}{{\`e}}1
}


\title{Interprétation de MiniJaja}
\begin{document}
\maketitle
Cette partie du rapport couvre le module MiniJajaInterpreter. L'interprétation du MiniJaja consiste à interpréter (ou exécuter) les instructions du code MiniJaja dans la mémoire. Nous prenons donc le code source MiniJaja en entrée, nous construisons ensuite l'arbre syntaxique correspondant et dans chaque nœud de cet AST nous implémentons les règles de l'interprétation. 
C'est-à-dire que nous effectuons les manipulations de la mémoire correspondant. 
Nous avons prévu deux politiques d'interprétation du MiniJaja : une interprétation \emph{atomique} du code source (utilisation nominale) et une interprétation \emph{pas-à-pas}(pour un mode debug par exemple). Le mode \emph{pas-à-pas} permettant de consulter l'état de la mémoire après l'exécution de chaque déclaration ou instruction. 

Dans cette partie nous évoquerons donc plus en détail l'implémentation de l'interprétation de MiniJaja et la technique choisie pour l'interprétation \emph{pas-à-pas}. 
Ensuite nous développerons le sujet des jeux de tests. 

\section{Interprétation MiniJaja}

L'interpréteur de MiniJaja est implémenté par la classe \emph{MiniJajaInterpreter}. C'est un visiteur (\emph{DefaultMiniJajaVisitor}) implémentant, pour chaque nœud, les différentes règles d'interprétation. On y retrouve donc :

\begin{itemize}
\item la déclaration en mémoire des variables et méthodes,
\item la mise à jour de la mémoire par les instructions,
\item le retrait des déclarations.
\end{itemize}

Déclarations et instructions nécessitent l'évaluation d'expressions. C'est la classe \emph{ExpressionEvaluator} qui a cette responsabilité, en s'appuyant sur la mémoire pour connaitre les valeurs des différentes variables. 
Le retrait des déclaration fait également l'objet d'une classe dédiée, \emph{RevokeInterpreter} qui implémente le visiteur par défaut de MiniJaja.
Le choix de classes séparées permet de partager les responsabilités et rend le code plus compréhensible, plus facilement maintenable et donc plus facilement testable.

\section{Politique d'exécution}  
Comme mentionné ci-dessus nous avons implémenté deux politiques \linebreak d'exécution du MiniJaja :\emph{atomique} et \emph{pas-à-pas}. 
Le première interprète le code de manière en une seule fois sans interruption (mode \emph{Run} d'un IDE par exemple).
La seconde interprète le code pas à pas, en stoppant après chaque déclaration et chaque assignation (mode \emph{Debug} d'un IDE par exemple. Pour cela la classe \emph{MiniJajaInterpreter} implémente la classe native \emph{Runnable} et donne accès aux méthodes :
\begin{itemize}
\item run() pour lancer l'interprétation.
\item resume() pour reprendre l'interprétation après une pause. La pause est gérée en privé dans la classe.
\item terminate() pour reprendre et compléter l'interprétation sans interruption.
\end{itemize}
C'est le module dédié à l'IHM qui aura la responsabilité d'exécuter ces méthodes dans un \emph{Thread} dédié à l'interprétation.

\section{Traitement des erreurs}
Il faut noter que l'interprétation de MiniJaja ne gère pas d'erreurs, car elle prévoit de recevoir en entrée un arbre de syntaxe abstrait correctement typé.
De plus, l'interprétation se base sur l'API de la mémoire qui elle gère les erreurs \emph{runtime} (par exemple le dépassement de tableau) par des exceptions qui sont affichées par l'interface graphique.

\section{Tests}
Les tests unitaires de la classe \emph{MiniJajaInterpreterTest} valident l'interprétation en vérifiant les différents états de la mémoire.
On retrouve des tests pour les deux politiques d'exécution citées plus haut. L'écriture des tests unitaires diffèrent ainsi :

\subsection{Tests unitaires en mode \emph{atomique}}

En mode \emph{atomique}, l'interprétation s'effectue en une fois sans interruption. Or en fin d'interprétation, la mémoire est vidée (suite eu retrait des déclarations). Pour constater les évolutions de la mémoire, on a décidé d'utiliser le librairie de test \emph{Mockito} qui permet, entre autres, de vérifier les appels de méthodes effectués pendant l'interprétation. La figure \ref{test_atomic} donne un exemple commenté de test unitaire d'interprétation d'une classe MiniJaja en mode \emph{atomique}.


\begin{figure}[!ht]
\begin{lstlisting}
@Test
public void interpretation_atomic() throws ParseException {
    //on parse un exemple de classe
    final ClassDeclaration declaration = parseClass(
            "class C {" +
            "   main {" +
            "      int var = 2;" +
            "      var = 3 + var;" +
            '}' +
            '}');
    //on instrumente la mémoire avec Mockito.spy pour effectuer plus tard des vérifications sur les appels
    final Memory memory = spy(new Memory());

    //on lance l'interprétation en mode "atomique" (sans pause)
    new MiniJajaInterpreter(memory, declaration, ATOMIC).run();

    //on utilise une instance de la classe Mockito InOrder pour vérifier des appels dans l'ordre de leurs exécutions
    final InOrder inOrder = inOrder(memory);

    //on vérifie la déclaration mémoire de la classe C puis de la variable var (initialisée à 2)
    inOrder.verify(memory).declVar(eq("C"), eq(UNDEFINED), eq(UNDEFINED_VALUE));
    inOrder.verify(memory).declVar(eq("var"), eq(INTEGER), argThat(isInteger(2)));

    //on vérifie ensuite l'affectation à 5 de la variable var (2+3)
    inOrder.verify(memory).affecterVal(eq("var"), argThat(isInteger(5)));

    //on vérifie en fin d'interprétation le retrait des déclarations (variable puis classe)
    inOrder.verify(memory).retirerDecl(eq("var"));
    inOrder.verify(memory).retirerDecl(eq("C"));

    //enfin, on vérifie qu'il n'y a pas eu d'autres appels à la mémoire que ceux vérifiés ci-dessus
    inOrder.verifyNoMoreInteractions();
}
\end{lstlisting}
\caption{Exemple de test en mode \emph{atomique}}
\label{test_atomic}
\end{figure}


\subsection{Tests unitaires en mode \emph{pas-à-pas}}

Le mode \emph{pas à pas} est le mode utilisé par l'interface graphique du projet. Pour le tester, il faut donc isoler l'exécution de l'interprétation de l'exécution du reste test (les vérifications). 
Le thread de test doit donc "attendre" l'exécution d'un pas d'interprétation avant de faire la vérification de la mémoire (mise en pause du thread par la méthode \emph{sleep}).
Après chaque vérification de la mémoire, on reprend l'interprétation grâce à la méthode \emph{resume}.
La figure \ref{test_stepByStep} donne un exemple commenté de test unitaire d'interprétation d'une classe MiniJaja en mode \emph{pas à pas} (basé sur la même classe MiniJaja que l'exmple de la figure \ref{test_atomic}).

\begin{figure}[!ht]
\begin{lstlisting}
@Test
public void interpretation_stepByStep() throws ParseException, InterruptedException {
    //on parse un exemple de classe
    final ClassDeclaration declaration = parseClass(
            "class C {" +
            "   main {" +
            "      int var = 2;" +
            "      var = 3 + var;" +
            '}' +
            '}');

    //on crée un interprétateur avec une mémoire. L'exécution de l'interprétation se fait dans un Thread séparé
    final Memory memory = new Memory();
    final MiniJajaInterpreter interpreter = new MiniJajaInterpreter(memory, declaration, STEP_BY_STEP);
    final Thread thread = new Thread(interpreter);
    //on démarre l'interprétation
    thread.start();

    //on attend la première pause...
    sleep(PAUSE_DURATION + 1);
    //...et on vérifie que la variable prend la valeur 2
    assertThat(memory.val("var"), isInteger(2));

    //on reprend l'interprétation
    interpreter.resume();
    //on attend la seconde pause...
    sleep(PAUSE_DURATION + 1);
    //...et on vérifie que la variable prend la valeur 5
    assertThat(memory.val("var"), isInteger(5));

    //on termine l'interprétation
    interpreter.terminate();
    thread.join();
    //...et on vérifie que la mémoire est vide
    assertTrue(memory.getStack().isEmpty());
}
\end{lstlisting}
\caption{Exemple de test en mode \emph{pas à pas}}
\label{test_stepByStep}
\end{figure}
\end{document}
